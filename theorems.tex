\usepackage{xcolor}
\usepackage{tcolorbox}
\tcbuselibrary{theorems}
\usepackage{amsmath, amsthm}

% Define custom colors
\definecolor{defscol}{HTML}{2929AB} % For definitions
\definecolor{asumscol}{HTML}{ff9d88} % For Assumptions
\definecolor{rmkscol}{HTML}{81F7D8} % For remarks
\definecolor{exmscol}{HTML}{888fff} % For examples
\definecolor{algscol}{HTML}{888fff} % For Algorithms
\definecolor{lemscol}{HTML}{88ffbe} % For Lemmas
\definecolor{thmscol}{HTML}{27B0F5} % For Theorems
\definecolor{prpscol}{HTML}{88ffbe} % For Propositions
\definecolor{corscol}{HTML}{88ffbe} % For Corollaries
\definecolor{clmscol}{HTML}{165c58} % For Claims
\definecolor{facscol}{HTML}{28a8a1} % For Facts

% ============================
% Definition
% ============================
\newtcbtheorem[number within=section]{mydefinition}{Definition}
{
    enhanced,
    frame hidden,
    titlerule=0mm,
    toptitle=1mm,
    bottomtitle=1mm,
    fonttitle=\bfseries\large,
    coltitle=black,
    colbacktitle=defscol!100!white,
    colback=defscol!15!white,
    boxrule=0pt, sharp corners, % Prevent overlap by removing sharp borders
    before skip=10pt, after skip=10pt % Add spacing around the box
}{defn}

\NewDocumentCommand{\defn}{m+m}{
    \begin{mydefinition}{#1}{}
        #2
    \end{mydefinition}
}

\NewDocumentCommand{\defnr}{mm+m}{
    \begin{mydefinition}{#1}{#2}
        #3
    \end{mydefinition}
}

% ============================
% Theorem
% ============================
\newtcbtheorem[use counter from=mydefinition]{mytheorem}{Theorem}
{
    enhanced,
    frame hidden,
    titlerule=0mm,
    toptitle=1mm,
    bottomtitle=1mm,
    fonttitle=\bfseries\large,
    coltitle=black,
    colbacktitle=thmscol!100!white,
    colback=thmscol!15!white,
    boxrule=0pt, sharp corners,
    before skip=10pt, after skip=10pt
}{thm}

\NewDocumentCommand{\thm}{m+m}{
    \begin{mytheorem}{#1}{}
        #2
    \end{mytheorem}
}

\NewDocumentCommand{\thmr}{mm+m}{
    \begin{mytheorem}{#1}{#2}
        #3
    \end{mytheorem}
}

% ============================
% Lemma
% ============================
\newtcbtheorem[use counter from=mydefinition]{mylemma}{Lemma}
{
    enhanced,
    frame hidden,
    titlerule=0mm,
    toptitle=1mm,
    bottomtitle=1mm,
    fonttitle=\bfseries\large,
    coltitle=black,
    colbacktitle=lemscol!40!white,
    colback=lemscol!20!white,
    boxrule=0pt, sharp corners,
    before skip=10pt, after skip=10pt
}{lem}

\NewDocumentCommand{\lem}{m+m}{
    \begin{mylemma}{#1}{}
        #2
    \end{mylemma}
}

% ============================
% Corollary
% ============================
\newtcbtheorem[use counter from=mydefinition]{mycorollary}{Corollary}
{
    enhanced,
    frame hidden,
    titlerule=0mm,
    toptitle=1mm,
    bottomtitle=1mm,
    fonttitle=\bfseries\large,
    coltitle=black,
    colbacktitle=corscol!40!white,
    colback=corscol!20!white,
    boxrule=0pt, sharp corners,
    before skip=10pt, after skip=10pt
}{cor}

\NewDocumentCommand{\cor}{+m}{
    \begin{mycorollary}{}{}
        #1
    \end{mycorollary}
}

% ============================
% Proposition
% ============================
\newtcbtheorem[use counter from=mydefinition]{myproposition}{Proposition}
{
    enhanced,
    frame hidden,
    titlerule=0mm,
    toptitle=1mm,
    bottomtitle=1mm,
    fonttitle=\bfseries\large,
    coltitle=black,
    colbacktitle=prpscol!30!white,
    colback=prpscol!20!white,
    boxrule=0pt, sharp corners,
    before skip=10pt, after skip=10pt
}{prop}

\NewDocumentCommand{\prop}{+m}{
    \begin{myproposition}{}{}
        #1
    \end{myproposition}
}

% ============================
% Claim
% ============================
\newtcbtheorem[use counter from=mydefinition]{myclaim}{Claim}
{
    enhanced,
    frame hidden,
    titlerule=0mm,
    toptitle=1mm,
    bottomtitle=1mm,
    fonttitle=\bfseries\large,
    coltitle=black,
    colbacktitle=clmscol!40!white,
    colback=clmscol!20!white,
    boxrule=0pt, sharp corners,
    before skip=10pt, after skip=10pt
}{clm}

\NewDocumentCommand{\clm}{m+m}{
    \begin{myclaim}{#1}{}
        #2
    \end{myclaim}
}

% ============================
% Fact
% ============================
\newtcbtheorem[use counter from=mydefinition]{myfact}{Fact}
{
    enhanced,
    frame hidden,
    titlerule=0mm,
    toptitle=1mm,
    bottomtitle=1mm,
    fonttitle=\bfseries\large,
    coltitle=black,
    colbacktitle=facscol!40!white,
    colback=facscol!20!white,
    boxrule=0pt, sharp corners,
    before skip=10pt, after skip=10pt
}{fact}

\NewDocumentCommand{\fact}{+m}{
    \begin{myfact}{}{}
        #1
    \end{myfact}
}

% ============================
% Proof
% ============================
\NewDocumentCommand{\pf}{+m}{
    \begin{proof}
        [\noindent\textbf{Proof.}]
        #1
    \end{proof}
}

% ============================
% Example
% ============================
\newtcbtheorem[number within=section]{myexample}{Example}
{
    enhanced,
    frame hidden,
    titlerule=0mm,
    toptitle=1mm,
    bottomtitle=1mm,
    fonttitle=\bfseries\large,
    coltitle=black,
    colbacktitle=exmscol!100!white,
    colback=exmscol!15!white,
    boxrule=0pt, sharp corners,
    before skip=10pt, after skip=10pt
}{exm}

\NewDocumentCommand{\exm}{m+m}{
    \begin{myexample}{#1}{}
        #2
    \end{myexample}
}

% ============================
% Remark
% ============================
\newtcbtheorem[number within=section]{myremark}{Remark}
{
    enhanced,
    frame hidden,
    titlerule=0mm,
    toptitle=1mm,
    bottomtitle=1mm,
    fonttitle=\bfseries\large,
    coltitle=black,
    colbacktitle=rmkscol!100!white,
    colback=rmkscol!15!white,
    boxrule=0pt, sharp corners,
    before skip=10pt, after skip=10pt
}{rmk}

\NewDocumentCommand{\rmkb}{+m}{
    \begin{myremark}{}{}
        #1
    \end{myremark}
}